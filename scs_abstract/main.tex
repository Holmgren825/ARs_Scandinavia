\documentclass[11pt]{article}

\usepackage{geometry}
\geometry{
  a4paper,
  left=25mm,
  right=25mm,
  top=25mm,
  }

\usepackage{fontspec}
\setmainfont{texgyretermes}[
   Extension = .otf,
   UprightFont = *-regular,
   BoldFont = *-bold,
   ItalicFont = *-italic,
   BoldItalicFont = *-bolditalic,
]
\usepackage{hyperref}
\hypersetup{
    colorlinks=false,
    linkcolor=blue,
    filecolor=magenta,
    urlcolor=cyan,
    hidelinks=true
    }


\begin{document}
\pagenumbering{gobble}
\begin{flushright}
	Swedish Climate Symposium 2024
\end{flushright}

\begin{center}
	\bf{Spatio-temporal characteristics of atmospheric rivers over Scandinavia}
\end{center}
Erik Holmgren$^{1,2}$, Hans Chen$^1$

\bigskip
\noindent
$^1$Department of Space, Earth, and Environment, Chalmers University of Technology, Gothenburg, Sweden

\noindent
$^2$\href{mailto:erik.holmgren@chalmers.se}{erik.holmgren@chalmers.se}

\bigskip
\noindent
Keywords: Atmospheric rivers, tropospheric moisture transport, regional climate

\bigskip
\noindent
In this poster, we will show some early results from our current study focusing on atmospheric rivers over Scandinavia.
We have used a state-of-the-art detection and tracking algorithm on ERA5 reanalysis data to identify, and compute key characteristics of, atmospheric rivers that make landfall over Scandinavia.
Atmospheric rivers are long and narrow filaments with unusually high quantities of tropospheric moisture transport, as measured by the Integrated Water Vapour Transport.
Globally, much of the poleward moisture transport occurs within atmospheric rivers.
Previously, impacts on the regional climate and circulation of atmospheric rivers have been extensively studied in both North America and Eastern Asia, where they have shown to play an important role in the variability of regional climate.
For instance, atmospheric rivers are known to play a significant role in extreme precipitation events in many locations around the world.
With this work, we aim to further examine how atmospheric rivers affect the Scandinavian climate through the evaluation of their intensity, frequency, seasonal variability and spatial patterns.


% Atmospheric rivers (ARs) are long, narrow bands of water vapor that transport moisture from the tropics to higher latitudes. They are known to cause extreme precipitation events and flooding in many parts of the world. In this study, we present early results from an investigation of ARs over Scandinavia. We use a combination of satellite data, reanalysis products, and ground-based observations to identify and track ARs over the region. Our preliminary findings suggest that ARs are a common feature of the Scandinavian climate, with a peak in activity during the winter months. We also find evidence of a strong relationship between ARs and extreme precipitation events in the region. Our ongoing work aims to further characterize the spatiotemporal variability of ARs over Scandinavia and to investigate their potential impacts on hydrological extremes. These results have important implications for regional water resources management and flood risk assessment

\bigskip
\noindent
Session: Atmospheric Circulation and Climate Dynamics — Session 2A

\noindent
Format: Poster

\bigskip
\noindent
\emph{Consent: The presenting author acts on behalf of and with the consent of all authors of this contribution.}

\end{document}
